\documentclass{ctexart}
\usepackage{amsmath}
\usepackage{xcolor}
\begin{document}
\title{概率论与数理统计}
\author{daydalek}
\date{\today}
\maketitle
\section{基本概念}
\subsection{随机事件}
\subsubsection{随机试验}
随机试验的要素:
\begin{itemize}
	\item 可以在相同的条件下重复进行
	\item 每次实验的结果可能不止一个,并且可以事先明确所有可能结果
	\item 进行一次试验之前不能确定哪个结果会出现
\end{itemize}
\subsubsection{样本空间}
随机试验E的所有可能结果集合称为\textbf{样本空间}\\
E的每个结果称为\textbf{样本点}\\
\subsubsection{随机事件}
E的样本空间S中的部分样本点组成的集合为E的\textbf{随机事件},简称为事件.\\
S中所有样本点的集合称为\textbf{必然事件}\\
不含任何样本点的集合称为\textbf{不可能事件}\\
\subsubsection{事件之间的关系运算}
\begin{itemize}
	\item 事件的包含与相等
	      \begin{equation}
		      (A\subseteq B) \land (B\subseteq A) \Rightarrow A=B
	      \end{equation}
	\item 事件的并运算
	      \begin{equation}
		      A\cup B=\{x|x\in A\lor x\in B\}
	      \end{equation}
	\item 事件的交运算
	      \begin{equation}
		      A\cap B=\{x|x\in A\land x\in B\}
	      \end{equation}
	\item 事件的差运算
	      \begin{equation}
		      A-B=\{x|x\in A\land x\notin B\}
	      \end{equation}
	\item 事件的对立事件
	      \begin{equation}
		      A^c=\{x|x\notin A\}
	      \end{equation}
	\item 事件的互不相容
	      \begin{equation}
		      A\cap B=\emptyset
	      \end{equation}
\end{itemize}
\section{频率和概率}
\subsection{频率}
\subsubsection{频率的定义}
设E是一个随机试验,其样本空间为S,事件A是E的一个随机事件,则事件A发生的频率为
\begin{equation}
	f_n(A)=\frac{n_A}{n}
\end{equation}
其中$n_A$是事件A发生的次数(频数),$n$是试验E进行的次数.\\
\subsubsection{概率}
对于E的每一个事件A,都有一个实数p(A)称为事件A的\textbf{概率},集合P满足
\begin{itemize}
	\item 非负性
	\item 规范性
	\item 可加性
\end{itemize}
概率有以下性质
\begin{itemize}
	\item $P(\emptyset)=0$
	\item 有限可加性
	\item $P(A)\leq 1$
	\item $A \subset B\Rightarrow P(B-A)=P(B)-P(A)$
\end{itemize}
广义加法公式(容斥原理)
\begin{equation}
    P(A\cup B)=P(A)+P(B)-P(A\cap B)
\end{equation}
它的推广:
\begin{equation}
    P(\bigcup_{i=1}^nA_i)=\sum_{i=1}^nP(A_i)-\sum_{i=1}^nP(A_i\cap A_j)+\sum_{i=1}^nP(A_i\cap A_j\cap A_k)-\cdots
\end{equation}
\subsection{等可能概型}
\subsubsection{古典概型}
\begin{itemize}
    \item 试验E的样本空间S是一个有限集合
    \item 试验E的每个样本点都是等可能的
\end{itemize}
\begin{equation}
    P(A)=\frac{k}{n}=\sum_{j=1}^kP({e_{ij}})
\end{equation}
k代表A中所含的基本事件数,而n代表样本空间S中的样本点数.\\
\
\end{document}