\documentclass{ctexart}
\usepackage{listing}
\usepackage{amsmath}


\begin{document}
\title{操作系统}
\author{daydalek}
\date{\today}
\maketitle

\section{计算机系统概述}
\subsection{操作系统的基本概念}
计算机系统的层次(上到下):
\begin{itemize}
    \item 用户
    \item 应用程序
    \item 操作系统
    \item 硬件
\end {itemize}
操作系统(operating system)是指控制和管理整个计算机系统的硬件与软件资源,合理地组织,调度计算机的工作和资源的分配,进而为用户和其他软件提供方便接口与环境的程序集合.\\
操作系统是计算机系统中\textbf{最基本的}系统软件
\subsection{操作系统的特征}
\begin{itemize}
    \item 并发(Concurrent):两个或多个事件在同一时间间隔内发生,是宏观而论的一种概念.操作系统引入\textbf{进程}的概念就是为了使程序能够并发执行.
    操作系统的并发是通过\textbf{分时}实现的.
    \item 共享(Sharing):系统中的资源可供内存中多个并发执行的进程共同使用.有以下两种共享方式:\\
    1.互斥共享:资源在一段时间内只允许一个进程使用.\\
    2.同时共享:资源可供多个进程"同时"使用,这里的同时是宏观上的概念,把一个请求分几个时间片段间隔地完成.
    \item 虚拟(Virtual):把一个物理上的实体变为若干个逻辑上的对应物\\
    操作系统的虚拟技术可以归纳为:时分复用和空分复用.
    \item 异步(Asynchronous):指进程的执行不是一贯到底,而是走走停停,以不可知的速度向前推进.
\end{itemize}
并发和共享是操作系统两个\textbf{最基本的特征}.
\subsection{操作系统的目标和功能}
操作系统应具有以下功能:
\subsubsection{资源的管理者}
\begin{itemize}
    \item 处理机管理:可以归纳为进程管理,包括进程控制,同步,通信,死锁处理,处理机调度等
    \item 存储器管理:包括内存分配,回收,内存保护和共享,地址映射,内存扩充等
    \item 设备管理:完成用户的i/o请求,方便用户使用各种设备,提高设备的利用率.包括缓冲管理,设备分配,设备处理和虚拟设备等.
    \item 文件管理:包括文件存储空间的管理,目录管理及文件读写管理和保护等.
\end{itemize}
\subsubsection{用户的接口}
\begin{itemize}
    \item 命令接口:用户通过命令解释器向操作系统发出命令,操作系统解释并执行命令,强调交互性\\
    分为1.联机命令接口:用户可以随时向操作系统发出命令,操作系统立即执行命令.\\
    2.脱机命令接口(批处理):用户将一组命令集中起来,一次性提交给操作系统,操作系统将这组命令集中的命令按照一定的顺序执行.
    \item 程序接口:用户可以通过编写程序来调用操作系统的服务.
\end{itemize}

\end{document}