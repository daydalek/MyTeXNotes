\documentclass{ctexart}
\usepackage{listing}
\usepackage{amsmath}

\begin{document}
\title{操作系统}
\author{daydalek}
\date{\today}
\maketitle

\section{计算机系统概述}
\subsection{操作系统的基本概念}
计算机系统的层次(上到下):
\begin{itemize}
\item 用户
\item 应用程序
\item 操作系统
\item 硬件
\end {itemize}
操作系统(operating system)是指控制和管理整个计算机系统的硬件与软件资源,合理地组织,调度计算机的工作和资源的分配,进而为用户和其他软件提供方便接口与环境的程序集合.\\
操作系统是计算机系统中\textbf{最基本的}系统软件
\subsection{操作系统的特征}
\begin{itemize}
	\item 并发(Concurrent):两个或多个事件在同一时间间隔内发生,是宏观而论的一种概念.操作系统引入\textbf{进程}的概念就是为了使程序能够并发执行.
	      操作系统的并发是通过\textbf{分时}实现的.
	\item 共享(Sharing):系统中的资源可供内存中多个并发执行的进程共同使用.有以下两种共享方式:\\
	      \begin{itemize}
		      \item 互斥共享:资源在一段时间内只允许一个进程使用.
		      \item 同时共享:资源可供多个进程"同时"使用,这里的同时是宏观上的概念,把一个请求分几个时间片段间隔地完成.
	      \end{itemize}
	\item 虚拟(Virtual):把一个物理上的实体变为若干个逻辑上的对应物\\
	      操作系统的虚拟技术可以归纳为:
	      \begin{itemize}
		      \item 时分复用
		      \item 空分复用
	      \end{itemize}
	\item 异步(Asynchronous):指进程的执行不是一贯到底,而是走走停停,以不可知的速度向前推进.
\end{itemize}
并发和共享是操作系统两个\textbf{最基本的特征}.
\subsection{操作系统的目标和功能}
操作系统应具有以下功能:
\subsubsection{资源的管理者}
\begin{itemize}
	\item 处理机管理:可以归纳为进程管理,包括进程控制,同步,通信,死锁处理,处理机调度等
	\item 存储器管理:包括内存分配,回收,内存保护和共享,地址映射,内存扩充等
	\item 设备管理:完成用户的i/o请求,方便用户使用各种设备,提高设备的利用率.包括缓冲管理,设备分配,设备处理和虚拟设备等.
	\item 文件管理:包括文件存储空间的管理,目录管理及文件读写管理和保护等.
\end{itemize}
\subsubsection{用户的接口}
\begin{itemize}
	\item 命令接口:用户通过命令解释器向操作系统发出命令,操作系统解释并执行命令,强调交互性\\
	      \begin{itemize}
		      \item 联机命令接口:用户可以随时向操作系统发出命令,操作系统立即执行命令.\\
		      \item 脱机命令接口(批处理):用户将一组命令集中起来,一次性提交给操作系统,操作系统将这组命令集中的命令按照一定的顺序执行.
	      \end{itemize}
	\item 程序接口:用户可以通过编写程序来调用操作系统的服务.
\end{itemize}
\subsubsection{操作系统用作扩充机器}
没有任何软件支持的计算机称为\textbf{裸机}\\
通常把覆盖了软件的机器称为\textbf{扩充机器}或者\textbf{虚拟机}
\subsection{计算机系统的发展和分类}
计算机操作系统分为:
\begin{itemize}
	\item 分时操作系统:把处理器的运行时间分成很小的时间片,若一个时间片内不能完成计算则暂停运行,处理器让给
	      其他程序使用,这就是\textbf{时间片轮转}\\
	      分时操作系统的主要特征:
	      \begin{itemize}
		      \item 同时性:允许多个终端用户同时使用
		      \item 交互性:用户能够方便地进行人机对话
		      \item 独立性:多个用户可以同时操作而不彼此干扰
		      \item 及时性:用户请求能在很短的时间内得到响应
	      \end{itemize}
	\item 实时操作系统
	      分为硬实时和软实时,能够抢占式调度,保证系统在规定的时间内完成任务.
	      实时操作系统的主要特征:
	      \begin{itemize}
		      \item 及时性
		      \item 可靠性
	      \end{itemize}
	\item 批处理操作系统:分为单道批处理系统和多道批处理系统
	      单道批处理系统的特点:
	      \begin{itemize}
		      \item 自动性:程序的执行无需人工干预
		      \item 顺序性:各道作业程序顺序地进入内存,完成顺序也相同(FIFO)
		      \item 单道性:内存中仅有一道程序运行
	      \end{itemize}
	      多道批处理系统的特点:
	      \begin{itemize}
		      \item 多道:计算机内存内同时存放多道程序
		      \item 宏观上并行:同时进入内存的多个程序都在运行中,
		      \item 微观上串行:多道程序轮流占有cpu,轮流执行.
	      \end{itemize}
	      多道批处理系统的优点是资源利用率高.
\end{itemize}
\subsection{操作系统的运行环境}
\subsubsection{操作系统的运行机制}
大多数操作系统的内核包含以下内容
\begin{itemize}
	\item 时钟管理
	\item 中断机制
	\item 原语(Atomic Operation):
	      \begin{itemize}
		      \item 处于操作系统最底层,最接近硬件
		      \item 具有原子性,操作只能一气呵成
		      \item 运行时间较短,调用频繁
	      \end{itemize}
	\item 系统控制的数据结构及处理:常见的操作有
	      \begin{itemize}
		      \item 进程管理
		      \item 存储器管理
		      \item 设备管理
	      \end{itemize}
\end{itemize}

\subsection{中断和异常的概念}
\begin{itemize}
	\item 中断(外中断):来自于cpu执行指令以外的事件发生,如外部设备的请求,时间片已到等.
	\item 异常(内中断,陷入):cpu执行指令内部的事件发生,例如非法操作码,地址越界等
\end{itemize}
\subsubsection{中断处理的过程}
\begin{itemize}
	\item 关中断
	\item 保存断点:为返回原来的地址,必须保存断点(即PC)
	\item 中断服务程序寻址:实质是取出中断服务程序的入口地址送入PC
	\item 保存现场和屏蔽字
	\item 开中断:允许更高级中断程序得到响应(因为已经保存完现场了)
	\item 执行中断服务程序:中断请求目的
	\item 关中断:保证在恢复现场和屏蔽字时不被更高级的中断程序打扰
	\item 恢复现场和屏蔽字
	\item 开中断,中断返回:返回到源程序的断点处,继续执行.
\end{itemize}
\subsection{系统调用}
指用户调用一些系统提供的子功能,按功能可以分为
\begin{itemize}
	\item 设备管理
	\item 文件管理
	\item 进程控制
	\item 进程通信
	\item 内存管理
\end{itemize}
这些处理需要内核负责完成.用户可以执行\textbf{陷入指令}(访管指令,trap指令)发起系统调用,请求系统提供服务,此时CPU从\textbf{用户态}
进入\textbf{核心态}.
下面是一些用户态转向核心态的例子
\begin{itemize}
	\item 用户要求系统调用
	\item 发生了一次中断
	\item 用户程序中产生了一个错误状态(异常)
	\item 用户程序企图执行一条特权指令
	\item 从核心态转向用户态(实现这个的指令本身也是特权命令,一般是中断返回命令)
\end{itemize}
访管指令是在用户态用的,因此它不是特权指令.
\subsection{操作系统的体系架构}
\subsubsection{大内核和微内核}
\begin{itemize}
	\item 大内核:操作系统的主要功能作为紧密整体运行在核心态
	\item 微内核:把内核中最基本的功能保留在核心态,其他转移到用户态,被移出内核的操作系统代码按照层次被划分为若干服务程序,执行相互独立,通信借由微内核进行\\
	      微内核的问题在于因为需要频繁切换用户态和核心态,执行开销偏大.有人提出把系统服务作为运行库链接到用户程序,称为\textbf{库操作系统}.
\end{itemize}

\section{进程管理}
\subsection{进程与线程}
\subsubsection{进程的概念}
进程的概念是为更好地描述和控制程序的\textbf{并发}执行,实现操作系统的并发性和共享型而定义的.\\
为了使得并发执行的程序能独立运行,配置了一种称为\textbf{进程控制块(Process Control Block)}的数据结构.\\
\textbf{PCB是进程存在的唯一标志!}

\subsubsection{进程的特征}
进程的基本特征是对比单个程序的顺序执行提出的,也是对进程管理提出的基本要求.
\begin{itemize}
	\item 动态性:进程是动态地产生,变化和消亡的,动态性是进程地\textbf{基本特征}
	\item 并发性:进程的重要特征,也是操作系统的重要特征.
	\item 独立性:进程实体是一个能独立运行,独立获得资源和独立接受调度的\textbf{基本单位}.
	\item 异步性:进程具有执行的间断性,为此操作系统必须配备进程同步机制以避免执行结果的不可再现性.
	\item 结构性:进程实体是由\textbf{程序段,数据段和PCB(进程控制块)}构成的.
\end{itemize}
\subsubsection{进程的状态与转换}
\begin{itemize}
	\item 运行态:进程正在处理机上运行.
	\item 就绪态:进程获得了除处理机之外的一切资源,一旦获得处理机资源可以立即开始执行.
	\item 阻塞态:又名等待态,正在等待某一事件而暂停运行.
	\item 创建态:进程正在被创建,包括申请空白PCB,向PCB中写入管理和控制进程信息,系统为进程分配所需的运行时资源等.
	\item 结束态:进程正从系统中消失.
\end{itemize}
三种基本状态之间转换如下
\begin{itemize}
	\item:就绪态$\rightarrow$运行态:就绪态的进程被分派处理机时间片.
	\item 运行态$\rightarrow$就绪态:在运行态的进程因为时间片用完而被抢占,或者在可剥夺的操作系统中让给更高优先级的进程.
	\item 运行态$\rightarrow$阻塞态:在运行态的进程请求某一资源的使用或等待某一事件的发生.
	\item 阻塞态$\rightarrow$就绪态:在阻塞态的进程所请求的资源或事件已经准备好,可以被使用.
\end{itemize}
\subsubsection{进程控制}
\begin{itemize}
	\item 进程的创建:
	\begin{itemize}
		\item 1.申请空白PCB,为新进程分配一个唯一的进程标识符PID.
		\item 2.为进程分配资源,若资源不足,进入阻塞态,等待资源分配.
		\item 3.初始化PCB,包括标志信息,处理机状态信息,处理机控制信息,进程优先级等.
		\item 4.若进程就绪队列能够接纳,则新进程插入队列,等待被调度运行.
	\end{itemize}
	\item 进程的终止:
	\begin{itemize}
		\item 1.根据PID找到PCB,检索PCB读出进程状态.
		\item 2.若进程处于运行态,则将其立即终止并分配处理机资源给其他进程.
		\item 3.若该进程还有子孙进程,也全部终止.
		\item 4.该进程的所有资源归还给父进程(如有)或操作系统.
		\item 5.PCB从队列(链表)中删除.
	\end{itemize}
	\item 进程的阻塞和唤醒:\\
	\begin{itemize}
		\item 阻塞:
		\item 1.找到PID标识的PCB
		\item 2.将其从等待队列中移除,置为就绪态
		\item 把该PCB插入到相应事件的等待队列中,处理机资源分配给其他就绪进程
		\item 唤醒:
		\item 1.在该事件的等待队列中找到PCB
		\item 2.从等待队列中移除,置为就绪态
		\item 3.把该PCB插入到就绪队列中,等待调度
	\end{itemize}
	\item 进程切换:\\
	\begin{itemize}
		\item 1.保存处理机上下文,包括PC和其他寄存器
		\item 2.更新PCB信息
		\item 3.把进程的PCB移入相应的就绪,在某事件阻塞等队列
		\item 4.选择另一个事件执行,更新PCB
		\item 5.更新内存管理的数据结构
		\item 6.恢复上下文
	\end{itemize}
\end{itemize}
\subsubsection{进程的组织}
\begin{itemize}
	\item 进程控制块:是进程存在的\textbf{唯一标志}.其各部分包括:
	\begin{itemize}
		\item 进程描述信息:进程标识符和用户标识符
		\item 进程控制和管理信息:进程状态,进程优先级,代码运行入口地址,程序的外存地址,进入内存时间,处理机占用时间,信号量使用情况等
		\item 资源分配清单:代码段指针,数据段指针,堆栈指针,文件描述符,键盘,鼠标等
		\item 处理机相关信息:通用寄存器值,地址寄存器值,控制寄存器值,标志寄存器值,状态字等,以便该进程能从断点继续执行
	\end{itemize}
	\item 程序段:被进程调度程序调度到CPU上执行的代码段,多个进程可以运行同一个程序
	\item 数据段:对应的程序进行加工处理的原始数据或程序执行的中间数据或最终结果.
\end{itemize}

\end{document}