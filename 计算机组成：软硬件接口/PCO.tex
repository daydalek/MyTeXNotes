\documentclass{ctexart}
\usepackage{xcolor}
\usepackage{listings}
\usepackage{amsmath}
\lstset{basicstyle=\ttfamily\small, % 设置字体样式
    language=C++} % 设置语言
\lstset{
    numbers=left, 
    numberstyle= \tiny, 
    keywordstyle= \color{ blue!70},
    backgroundcolor= \color{white}, % 背景颜色
    frame=shadowbox, % 阴影效果
    rulesepcolor= \color{ red!20!green!20!blue!20},
    stringstyle= \color{red!50!green!50!blue!50},
    escapeinside=``, % 英文分号中可写入中文
    xleftmargin=2em, aboveskip=1em,
    framexleftmargin=2em
}

\begin{document}
\title{计算机组成:软硬件接口}
\author{daydalek}
\date{\today}
\maketitle
\section{概要}
\subsection{衡量计算机系统的速度}
如果时间来度量计算机的性能,则完成相同的计算任务,需要时间更少的计算机更快\\
使用CPU执行时间(CPU execution time) 它只表示CPU上花费的时间\\
使用系统性能(system performance)来表示空载系统的响应时间\\
并用属于CPU性能(CPU performance)的术语来表示用户CPU时间\\
\subsection{相关公式}
\begin{equation}
\mbox{程序的CPU执行时间}=\mbox{程序的CPU时钟周期数}\times\mbox{时钟周期时间}
\end{equation}
\begin{equation}
\mbox{程序的CPU执行时间}=\mbox{程序的CPU时钟周期数}/\mbox{时钟 频率}
\end{equation}
这是因为时钟周期时间和时钟频率是相互倒数的.
一个程序需要的时钟周期数可写为
\begin{equation}
\mbox{程序的CPU时钟周期数}=\mbox{程序的指令数}\times\mbox{每条指令的平均时钟周期数}
\end{equation}
术语CPI(Cycles Per Instruction)表示每条指令的平均时钟周期数\\
于是我们得到以下公式
\begin{equation}
\mbox{CPU时间}=\mbox{指令数}\times CPI \times\mbox{时钟周期时间}
\end{equation}
\end{document}


