\documentclass{ctexart}
\usepackage{xcolor}
\usepackage{listings}
\usepackage{amsmath}
\lstset{basicstyle=\ttfamily\small} % 设置字体样式 % 设置语言
\lstset{
    numbers=left, 
    numberstyle= \tiny, 
    keywordstyle= \color{ blue!70},
    backgroundcolor= \color{white}, % 背景颜色
    frame=shadowbox, % 阴影效果
    rulesepcolor= \color{ red!20!green!20!blue!20},
    stringstyle= \color{red!50!green!50!blue!50},
    escapeinside=``, % 英文分号中可写入中文
    xleftmargin=2em, aboveskip=1em,
    framexleftmargin=2em
}

\begin{document}
\title{计算机组成:软硬件接口}
\author{daydalek}
\date{\today}
\maketitle
\section{概要}
\subsection{衡量计算机系统的速度}
如果时间来度量计算机的性能,则完成相同的计算任务,需要时间更少的计算机更快\\
使用CPU执行时间(CPU execution time) 它只表示CPU上花费的时间\\
使用系统性能(system performance)来表示空载系统的响应时间\\
并用属于CPU性能(CPU performance)的术语来表示用户CPU时间\\
\subsection{相关公式}
\begin{equation}
\mbox{程序的CPU执行时间}=\mbox{程序的CPU时钟周期数}\times\mbox{时钟周期时间}
\end{equation}
\begin{equation}
\mbox{程序的CPU执行时间}=\mbox{程序的CPU时钟周期数}/\mbox{时钟 频率}
\end{equation}
这是因为时钟周期时间和时钟频率是相互倒数的.
一个程序需要的时钟周期数可写为
\begin{equation}
\mbox{程序的CPU时钟周期数}=\mbox{程序的指令数}\times\mbox{每条指令的平均时钟周期数}
\end{equation}
术语CPI(Cycles Per Instruction)表示每条指令的平均时钟周期数\\
于是我们得到以下公式
\begin{equation}
\mbox{CPU时间}=\mbox{指令数}\times CPI \times\mbox{时钟周期时间}
\end{equation}

\section{指令}
\subsection{计算机硬件的操作数}
MIPS体系架构中的一字为32位(MIPS32)或64位(MIPS64) 此处讨论MIPS32\\
\subsubsection{存储器操作数}
由于MIPS只能操作寄存器中的数据,因此需要包含在存储器和寄存器之间传递数据的指令,
这些指令叫做数据传送指令(Data Transfer Instructions)\\
为了访问存储器中的每一个字,指令需要给出存储器地址,将数据从
存储器复制到寄存器的指令叫做取数指令(Load Instructions),格式是lw+目标寄存器+
用于寻址的常数和寄存器\\
一个例子:\\
\begin{lstlisting}
    A[12]=h+A[8];
    -----------------------------------------------
    lw $t0,32($s3)  # $t0=A[8]
    add $t0,$t0,$s2 # $t0=h+A[8]
    sw $t0,48($s3)  # store $t0 in A[12]
\end{lstlisting}
\subsubsection{立即数操作数}
立即数操作数是指令中的常数,它们是在指令中直接给出的,而不是从存储器中读取的\\
例如要向S3中+4,可以这么写
\begin{lstlisting}
    addi $s3,$s3,4
\end{lstlisting}
相对从存储器提取常数,使用立即数操作更快\\
\subsection{计算机中指令的表示}
以这段代码为例
\begin{lstlisting}
    add $t0,$s1,$s2
\end{lstlisting}
其十进制表示为
0 \t 17 \t 18 \t 8 \t 0 \t 32 \\
以二进制表示是
000000 \t 01001 \t 01010 \t 00000 \t 01000 \t 100000 \\
指令的布局叫做指令格式(Instruction Format) 可以看出它是32位长的;
为了与汇编语言区分,称其为机器语言(Machine Language),指令序列称为机器码(Machine Code)\\
\subsubsection{MIPS字段}
为了简化,把以上字段命名如下\\
op \t rs \t rt \t rd \t shamt \t funct \\
其中op是操作码,rs是源1寄存器,rt是源2寄存器,rd是目的寄存器,shamt是移位量,funct是功能码\\
它们的位数分别是6,5,5,5,5,6\\
然而,这种安排存在着局限性,它使得寻址范围极为有限。
例如,取字指令必须指定两个
寄存器和一个常数 在上述格式中,如果地址使用其中的一个5位字段,那么取字指令的常数
就被限制在$2^5$(即32)之内.这个常数通常用来从数组或数据结构中选择元素,显然是不够的。
所以,MIPS还引入另一种指令格式,同样是32位长,但是它的布局是
op \t rs \t rt \t constant or address \\
这种格式的指令称为I型指令,它的字段分别是6,5,5,16\\
注意,此处的rt含义已经发生变化,它不再是源操作数寄存器,而是目的寄存器\\
举个例子,取字指令的格式是
\begin{lstlisting}
    lw $t0,32($s3)
\end{lstlisting}
它的机器码是
100011 \t 10011 \t 01000 \t 0000000000100000 \\
16位地址或常数码支持更大的基址偏移量,从而支持更大范围的寻址\\
this is a online test for vscode online 
\end{document}


