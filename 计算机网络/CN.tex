\documentclass{ctexart}
\usepackage{xcolor}
\usepackage{listings}
\usepackage{amsmath}
\lstset{basicstyle=\ttfamily\small} % 设置字体样式 % 设置语言
\lstset{
    numbers=left, 
    numberstyle= \tiny, 
    keywordstyle= \color{ blue!70},
    backgroundcolor= \color{white}, % 背景颜色
    frame=shadowbox, % 阴影效果
    rulesepcolor= \color{ red!20!green!20!blue!20},
    stringstyle= \color{red!50!green!50!blue!50},
    escapeinside=``, % 英文分号中可写入中文
    xleftmargin=2em, aboveskip=1em,
    framexleftmargin=2em}
\begin{document}
\title{计算机网络}
\author{daydalek}
\date{\today}
\maketitle
\section{计算机网络概述}
\subsection{计算机网络的定义}
计算机网络就是通过线路互连起来的,自治的计算机集合,确切地说是
将分布在不同地理位置上的具有独立工作能力地计算机、终端及其附属设备
用通信设备和通信线路连接起来,并配置网络软件,以实现计算机资源共享的
系统。\\
1969年,ARAPNET投入运行,这是互联网的雏形。
\subsection{互联网的组成}
互联网复杂的拓扑结构可以简化为\\
\begin{itemize}
    \item 核心部分
    \item 边缘部分
\end{itemize}
核心部分:由大量网络和连接这些网络的路由器组成,为边缘部分提供服务。\\
边缘部分:由所有连接在互联网上的主机组成。这部分是由用户直接使用的。\\
\subsubsection{边缘部分}
在边缘的端系统之间的通信方式可以划分为
\begin{itemize}
    \item 客户/服务器方式(C/S)
    \item 对等方式(P2P)
\end{itemize}
客户/服务器方式:客户端向服务器端发送请求,服务器端响应请求。\\
特点:客户是服务请求方,服务器是服务提供方。\\
客户程序在被用户调用后运行,在通信时主动向远地服务器发起通信,
故客户程序必须知道服务器程序的地址。\\
服务器程序是一种专门用来提供某种服务的程序,可同时处理多个远地或
本地客户的请求。\\
系统启动后即自动调用并一直不断地运行着,被动等待接受客户的通信
请求,因此服务器端不需要知道客户程序的地址。\\
对等方式:两个对等实体之间直接进行通信。\\
特点:本质上仍然是客户-服务器方式,只是每一台主机既是服务器又是客户。\\

\subsubsection{互联网的核心部分}
在网络核心部分起特殊作用的是
\begin{itemize}
    \item 路由器
\end{itemize}
路由器:一种专用计算机,是实现分组交换(packet switching)的关键设备。\\
路由器的任务是转发收到的分组。\\
交换的方式分为
\begin{itemize}
    \item 电路交换
    \item 分组交换
    \item 报文交换
\end{itemize}
电路交换:在两个通信实体之间建立一条专用的物理链路,然后在该链路上
进行通信。\\
电路交换的过程:建立连接——通话——释放连接。\\
由于通话时即使没有数据传输,通信资源也始终被占据,
因此电路交换的资源利用率较低。\\
分组交换:将数据分割成若干个数据包,然后将这些数据包传送到目的地。\\
分组转换的结构:
\begin{itemize}
    \item 报文:要发送的整块数据
    \item 首部:控制信息,也成为“包头”
    \item 分组:数据段(报文)前加上首部,也称为“包”
\end{itemize}
分组交换的优点:
\begin{itemize}
    \item 高效:动态分配带宽,逐段占用链路
    \item 灵活:以分组为单位选择最合适的转发路由
    \item 可靠:保证可靠性的网络协议,分布式多路由的分组交换网
    \item 迅速:以分组为传送单位,不事先建立连接就能发送分组
\end{itemize}
分组交换的缺点:
\begin{itemize}
    \item 时延:存储转发排队带来的
    \item 开销:控制信息带来了额外的开销
\end{itemize}
报文交换:自古就有的邮政通信便是一种报文交换,之后的电报也是。
它和分组交换的主要区别是不加分组而一次传送整条报文,存储后转发。
它们的应用场景:
\begin{itemize}
    \item 电路交换:连续传输大量的数据,传送时间远大于连接建立时间时
    \item 分组交换:传送突发数据
\end{itemize}

\section{计算机网络的分类}
按照作用范围:
\begin{itemize}
    \item WAN:广域网
    \item MAN:城域网
    \item LAN:局域网
    \item PAN:个人区域网
\end{itemize}
按照使用者:

\begin{itemize}
    \item 公用网
    \item 专用网
\end{itemize}

\section{计算机网络的性能}
7个常用的性能指标
\begin{itemize}
    \item 速率
    \item 带宽
    \item 吞吐量
    \item 时延
    \item 时延带宽积
    \item 往返时间RTT
    \item 利用率
\end{itemize}
\end{document}